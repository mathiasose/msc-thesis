\section{Motivation}
As mentioned in Section \ref{sec:finding_transitions},
using a more advanced encoding for CA tasks may enable a more advanced search algorithm.
This combination may then lead to successfully solving tasks that are considered difficult with the classical encoding and algorithm.

In fields such as neuroevolution,
NEAT has been shown to produce useful patterns,
without using temporal development and local interactions.
However, these tools \textit{are} used by nature in processes such as embryogenesis,
that we can model with CA.
In the quest to advance cellular systems towards biological levels of complexity,
it is worth investigating if this technique, which is proven in one field,
lends itself to being adapted as a part of a process in another field.

By testing a new model on a few select tasks, we may not only learn whether the model can accomplish the task,
but we may gain more general insights into the components (CA, the architecture; CPPN, the encoding; NEAT, the algorithm) that make up the model,
shedding further light on domains such as artificial life and morphogenetic engineering.
If something works well, we can try replacing a component with a new one to see if it still works,
or if something does not work well, we can try replacing a component to see if that changes anything.
