\section{Synchronization Problem}
Another problem for binary CA is called the \textit{synchronization problem}.
From some arbitrary initial configuration,
the CA should find its way to a two-step cyclic attractor where all cells share the same state in one timestep,
then all share the other state in the next timestep.
It is thus simillar to the majority problem in the way information must be transmitted across the CA in order to coordinate the cells,
but instead of having to "count" cells and landing in a specific point attractor, it finds a cyclic attractor without concern for which of the two states it visits first.

The fitness evaluation function for this experiment is based on the one described in \cite{das1995evolving}, but with some modifications.
$K=100$ random initial CA configurations are generated from a TODO distribution.
Each candidate solution is first tested on $I=25$ initial configurations.
If none of the $I$ configurations results in the desired behavior, a fitness of $0.0$ is reported.
If any of the configurations does result in the desired behavior, the candidate is tested on all $K$ configurations.
The fitness reported is then the ratio of configurations that show the correct behavior.

