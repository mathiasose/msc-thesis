\chapter{Introduction}
\textit{Cellular Automata} (CA) were first conceptualized and introduced in the 1940's and 1950's.
Around the same time, the idea of the \textit{evolutionary algorithm} was also being developed independently.
Both of these concepts take inspiration from nature, and thus fall into the category of \textit{artificial life}.
One goal of this field of research is to create systems that are of complexity comparable to that of biological systems found in nature.

One way to try to achieve this goal is to combine these two distinct concepts: Cellular systems designed by evolution.
Many different kinds of tasks have been solved by cellular systems that have been created this way.
However, more complex tasks and more complex models means greater spaces of possible solutions that the evolutionary algorithm must search.
%This phenomenon is seen in many fields, and is often referred to as "the curse of dimensionality" (TODO citation).
For a traditional genetic algorithm it can be both time consuming and challenging to find good solutions.

One of the possible ways to remedy this problem is to replace the traditional table-based encoding of CA transition rules with a different encoding: an encoding that supports a more complex evolutionary algorithm.
This paper describes the investigation of using \textit{Compositional Pattern Producing Networks} (CPPNs) as the data structure for transition rules,
and the \textit{NeuroEvolution of Augmenting Topologies} (NEAT) genetic algorithm for evolving these CPPNs.

With CPPN-based transition functions there is not a linear relationship between the input-output size and the size of the encoding.
The algorithm starts with the smallest possible encoding and iteratively over time adds features to it and adjusts them until an optimal solution is found.
This \textit{complexification} mimics the process that biologists believe life on earth developed.

To test this new combination, which we call \textit{CA-NEAT},
a custom Python framework was built to run simulations in software.
The framework was tasked with solving various CA tasks of different difficulties, with different degrees of success.

TODO research question

TODO outline thesis
